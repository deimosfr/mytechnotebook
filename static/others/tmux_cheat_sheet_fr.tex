% -----------------------------------------------------------------------
% Cheat Sheet Tmux
%
% Usage in Document content :
% \section : create a new section in column
% 
% \columnbreak : break the column to force following content to jump to
%                the next column
% \cm{command}{description} : set a dotted line between the command and
%                             the description
% -----------------------------------------------------------------------

% -----------------------------------------------------------------------
% Document settings
% -----------------------------------------------------------------------

\documentclass[10pt,landscape]{article}
\usepackage{multicol}
\usepackage{calc}
\usepackage{ifthen}
\usepackage[landscape]{geometry}
\usepackage{amsmath,amsthm,amsfonts,amssymb}
\usepackage{color,graphicx,overpic}
\usepackage{hyperref}
\usepackage{amsmath}

% PDF informations
\pdfinfo{
  /Title (tmux_cheat_sheet.pdf)
  /Creator (TeX)
  /Producer (pdfTeX 1.40.0)
  /Author (Pierre Mavro)
  /Subject (Tmux)
  /Keywords (pdflatex, latex,pdftex,tex)}

% This sets page margins to .5 inch if using letter paper, and to 1cm
% if using A4 paper. (This probably isn't strictly necessary.)
% If using another size paper, use default 1cm margins.
\ifthenelse{\lengthtest { \paperwidth = 11in}}
    { \geometry{top=.5in,left=.5in,right=.5in,bottom=.5in} }
    {\ifthenelse{ \lengthtest{ \paperwidth = 297mm}}
        {\geometry{top=1cm,left=1cm,right=1cm,bottom=1cm} }
        {\geometry{top=1cm,left=1cm,right=1cm,bottom=1cm} }
    }

% Turn off header and footer
\pagestyle{empty}

% Redefine section commands to use less space
\makeatletter
\renewcommand{\section}{\@startsection{section}{1}{0mm}%
                                {-1ex plus -.5ex minus -.2ex}%
                                {0.5ex plus .2ex}%x
                                {\normalfont\large\bfseries}}
\renewcommand{\subsection}{\@startsection{subsection}{2}{0mm}%
                                {-1explus -.5ex minus -.2ex}%
                                {0.5ex plus .2ex}%
                                {\normalfont\normalsize\bfseries}}
\renewcommand{\subsubsection}{\@startsection{subsubsection}{3}{0mm}%
                                {-1ex plus -.5ex minus -.2ex}%
                                {1ex plus .2ex}%
                                {\normalfont\small\bfseries}}
\makeatother

% Define BibTeX command
\def\BibTeX{{\rm B\kern-.05em{\sc i\kern-.025em b}\kern-.08em
    T\kern-.1667em\lower.7ex\hbox{E}\kern-.125emX}}

% Don't print section numbers
\setcounter{secnumdepth}{0}

% Set vertical view instead of horizontal (set to 0 to let it choose)
%\setcounter{unbalance}{45}
\setcounter{unbalance}{0}

\setlength{\parindent}{0pt}
\setlength{\parskip}{0pt plus 0.5ex}

%My Environments
\newtheorem{example}[section]{Example}

% Dot lines between command and description
\def\cm#1#2{{\tt#1}\dotfill#2\par}

% -----------------------------------------------------------------------
% Document start
% -----------------------------------------------------------------------

\begin{document}
\raggedright
\footnotesize
% Set number of columns
\begin{multicols}{3}

% Multicol parameters
% These lengths are set only within the two main columns
\setlength{\columnseprule}{0.25pt}
\setlength{\premulticols}{1pt}
\setlength{\postmulticols}{1pt}
\setlength{\multicolsep}{1pt}
\setlength{\columnsep}{2pt}

% -----------------------------------------------------------------------
% Document content
% -----------------------------------------------------------------------

\begin{center}
     \Large{\bf{Tmux Cheat Sheet}} \\
\end{center}

Toutes les fonctions list\'ees ci dessous doivent \^etre pr\'ec\'ed\'ee de la combinaison de touches Ctrl+b

\section{Commandes}
\cm{tmux}{Lancer tmux}
\cm{tmux a}{Se rattacher \`a une session}

\section{Gestion des fen\^etres}
\cm{c}{Cr\'eer une nouvelle fen\^etres}
\cm{w}{Obtenir la liste des fen\^etres ouvertes}
\cm{n}{Se d\'eplacer sur la fen\^etres suivante}
\cm{p}{Se d\'eplacer sur la fen\^etres pr\'ec\'edente}
\cm{l}{Se d\'eplacer sur l'avant derni\`ere fen\^etres utilis\'ee}
\cm{[0-9]}{Se d\'eplacer sur une fen\^etres en fonction de son num\'ero}
\cm{f \textit{nom}}{Faire une recherche dans les buffers des fen\^etres}
\cm{'}{Renommer la fen\^etres en cours}
\cm{\&}{Forcer la fermeture d'une fen\^etres}
\cm{t}{Afficher l'heure}
\columnbreak

\section{Split}
\cm{"}{D\'ecouper horizontalement l'\'ecran}
\cm{\%}{D\'ecouper verticalement l'\'ecran}
\cm{\{}{Se d\'eplacer sur le volet pr\'ec\'edent}
\cm{\} \textit{ou} o}{Se d\'eplacer sur le volet suivant}
\cm{$\leftarrow$ $\rightarrow$ $\uparrow$ $\downarrow$}{Se d\'eplacer sur le volet correspondant \`a la touche}
\cm{q}{Obtenir les num\'eros des volets}
\cm{$\sqcup$}{Changer l'organisation visuelle des volets}
\cm{Alt+($\leftarrow$ $\rightarrow$ $\uparrow$ $\downarrow$)}{Agrandir une r\'eduire un volet}
\cm{!}{Convertir un volet d'un split en fen\^etre}
\cm{:joinp -h -s 0.0 -p 75}{Convertir une fen\^etre pour int\'egration dans un split}
$\bullet$ -h : horizontalement \\
$\bullet$ -s 0.0 : fen\^etre 0 et volet 0\\
$\bullet$ -p 75 : occupation \`a 75\% de la fen\^etre


\section{L'historique}
\cm{$\Uparrow$ (PageUP)}{Remonter dans l'historique}
\cm{$\Downarrow$ (PageDOWN)}{Descendre dans l'historique apr\`es \^etre remont\'e}
\cm{$\sqcup$ \textit{puis}($\uparrow$ $\downarrow$)}{S\'electionner des lignes de l'historique (apr\`es $\Uparrow$)}
\cm{$\hookleftarrow$}{Copier la s\'election}
\cm{=}{Coller la s\'election}
\columnbreak

\section{Session}
\cm{d}{D\'etachement de session tmux}
\cm{s}{Lister les sessions tmux}
\cm{(}{Basculer \`a la session tmux suivante}
\cm{)}{Basculer \`a la session tmux pr\'ec\'edente}

% Autor
\rule{0.3\linewidth}{0.25pt}
\section{Auteur}
\href{mailto:pierre@mavro.fr}{Pierre Mavro (Deimosfr)} \\
\url{http://www.mavro.fr} $\cdot$ \url{http://www.deimos.fr}

% You can even have references
\rule{0.3\linewidth}{0.25pt}
\scriptsize
\bibliographystyle{abstract}
\bibliography{refFile}
\url{http://tmux.sourceforge.net/}
\url{http://myhumblecorner.wordpress.com/2011/08/30/screen-to-tmux-a-humble-quick-start-guide/}
\url{http://blog.hawkhost.com/2010/06/28/tmux-the-terminal-multiplexer/}
\url{http://blog.hawkhost.com/2010/07/02/tmux-\%E2\%80\%93-the-terminal-multiplexer-part-2/}
\url{http://www.dayid.org/os/notes/tm.html}
\end{multicols}
\end{document}
